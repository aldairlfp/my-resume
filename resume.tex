%!TEX TS-program = xelatex
%!TEX encoding = UTF-8 Unicode
% Awesome CV LaTeX Template
%
% This template has been downloaded from:
% https://github.com/posquit0/Awesome-CV
%
% Author:
% Claud D. Park <posquit0.bj@gmail.com>
% http://www.posquit0.com
%
% Template license:
% CC BY-SA 4.0 (https://creativecommons.org/licenses/by-sa/4.0/)
%


%%%%%%%%%%%%%%%%%%%%%%%%%%%%%%%%%%%%%%
%     Configuration
%%%%%%%%%%%%%%%%%%%%%%%%%%%%%%%%%%%%%%
%%% Themes: Awesome-CV
\documentclass[]{awesome-cv}
\usepackage{textcomp}
%%% Override a directory location for fonts(default: 'fonts/')
\fontdir[fonts/]

%%% Configure a directory location for sections
\newcommand*{\sectiondir}{resume/}

%%% Override color
% Awesome Colors: awesome-emerald, awesome-skyblue, awesome-red, awesome-pink, awesome-orange
%                 awesome-nephritis, awesome-concrete, awesome-darknight
%% Color for highlight
% Define your custom color if you don't like awesome colors
\colorlet{awesome}{awesome-red}
%\definecolor{awesome}{HTML}{CA63A8}
%% Colors for text
%\definecolor{darktext}{HTML}{414141}
%\definecolor{text}{HTML}{414141}
%\definecolor{graytext}{HTML}{414141}
%\definecolor{lighttext}{HTML}{414141}

%%% Override a separator for social informations in header(default: ' | ')
%\headersocialsep[\quad\textbar\quad]
\begin{document}

%%%%%%%%%%%%%%%%%%%%%%%%%%%%%%%%%%%%%%
%     Profile
%%%%%%%%%%%%%%%%%%%%%%%%%%%%%%%%%%%%%%
\begin{center}
	\headerfirstnamestyle{Aldair} \headerlastnamestyle{Alfonso} \\
	\vspace{2mm}
	{\hspace{0.8cm}\faEnvelope\ aldairalfonsoperez@gmail.com}  |  {\faMobile\ (+53) 58123551}  |  {\faMapMarker\ Bilbao, Spain}
	\newline {\faLink\ \href{https://www.linkedin.com/in/aldair-alfonso-722b5421a/}{https://www.linkedin.com/in/aldair-alfonso-722b5421a/}}
\end{center}

\cvsection{Skills}
\begin{cventries}
	\cventry
	{}
	{\def\arraystretch{1.15}{\begin{tabular}{ l l }
				Languages:                            & {\skill{ C\# (Advanced), Python, C, C++, SQL, JavaScript.}}                           \\
				Game Development:                     & {\skill{ Unity 3D (Learning), Game Logic Design, UI/UX Design, Network Programming.}} \\
				Frameworks:                           & {\skill{ .Net, ASP.NET, Django, FastAPI.}}                                            \\
				Databases:                            & {\skill{ MySQL, PostgreSQL, SQLite.}}                                                 \\
				Technologies / Tools: \hspace{0.05cm} & {\skill{ Visual Studio, Unity Editor, Git, Docker, npm.}}                             \\
				Practices:                            & {\skill{ Agile, Scrum, SOLID Principles, Test-Driven Development, Code Reviews.}}     \\
			\end{tabular}}}
	{}
	{}
	{}
\end{cventries}
\vspace{-7mm}
%%%%%%%%%%%%%%%%%%%%%%%%%%%%%%%%%%%%%%
%     Experience
%%%%%%%%%%%%%%%%%%%%%%%%%%%%%%%%%%%%%%
\cvsection{Experience}
\begin{cventries}
	\cventry
	{Data Engineer}
	{Avangenio}
	{Miami, USA}
	{10/2024 – 07/2025}
	{\begin{cvitems}
			\vspace{0.5mm}
			\item {Implemented optimized data processing pipelines using Python and pandas, reducing data transformation time by 40\% through efficient algorithm design and vectorized operations.}
			\item {Redesigned database schema architecture and optimized query performance with SQL and Python ORM, achieving 60\% faster data retrieval for critical business operations.}
			\item {Built automated data validation systems using Python that reduced data quality issues by 85\%, implementing comprehensive error handling and logging mechanisms for better system reliability.}
			\item {Developed real-time data visualization dashboards using Python frameworks (Dash/Plotly), enabling stakeholders to make data-driven decisions 50\% faster than previous manual reporting methods.}
		\end{cvitems}}

	\cventry
	{Software Engineer}
	{Knales}
	{Havana, Cuba}
	{10/2022 – 11/2022}
	{\begin{cvitems}
			\vspace{0.5mm}
			\item {Built a Telegram bot using .NET/C\# with automated message processing, reducing manual customer response time by 70\% and handling 500+ daily user interactions through efficient event-driven architecture.}
		\end{cvitems}}
\end{cventries}

\cvsection{Projects}
\begin{cventries}
	\vspace{-3mm}
	\cventry
	{}
	{Bruce \vspace{-5mm}}
	{Python \vspace{-5mm}}
	{}
	{\begin{cvsectionnormaltext}
			\item {Developed Bruce, a high-performance compiler for the Hulk Language, featuring an automata-based lexer for efficient
			            tokenization, an LL(1) parser generator utilizing a custom Grammar class for deterministic syntax analysis, and a robust
			            semantic analysis module to enforce language constraints and detect errors, ensuring reliable and efficient code
			            compilation
			            \newline \faLink\ \href{https://github.com/aldairlfp/bruce}{https://github.com/aldairlfp/bruce}}
		\end{cvsectionnormaltext}}

	\vspace{-3mm}
	\cventry
	{}
	{Manage Site \vspace{-5mm}}
	{Python, Django \vspace{-5mm}}
	{}
	{\begin{cvsectionnormaltext}
			\item{Developed a RESTful API with Django to manage organization members, salaries, and fee payments, featuring an
			            optimized database schema and caching for fast data retrieval. Skills in API design and real-time data management
			            directly applicable to multiplayer game backend systems and player data synchronization.}
		\end{cvsectionnormaltext}}

	\vspace{-3mm}
	\cventry
	{}
	{Awars \vspace{-5mm}}
	{Python \vspace{-5mm}}
	{}
	{\begin{cvsectionnormaltext}
			\item{Awars is a battle simulation system where units (agents) engage in combat scenarios. Features a Domain-Specific Language
			            for game logic control, demonstrating experience with combat mechanics and game rule implementation relevant to 1v1 combat systems.
			            \newline \faLink\ \href{https://github.com/aldairlfp/Awars}{https://github.com/aldairlfp/Awars}}
		\end{cvsectionnormaltext}}

	\vspace{-3mm}
	\cventry
	{}
	{Information Retrieval System \vspace{-5mm}}
	{Python \vspace{-5mm}}
	{}
	{\begin{cvsectionnormaltext}
			\item{The objective of this work is to carry out an information retrieval system that allows users to perform searches in a set
			            of documents, and obtain as a result a list of documents ordered by relevance.
			            \newline \faLink\ \href{https://github.com/aldairlfp/SRI}{https://github.com/aldairlfp/SRI}}
		\end{cvsectionnormaltext}}

	\vspace{-3mm}
	\cventry
	{}
	{DTorrent \vspace{-5mm}}
	{Python \vspace{-5mm}}
	{}
	{\begin{cvsectionnormaltext}
			\item{Torrent client and a distributed tracker, implementing peer-to-peer file sharing and decentralized metadata exchange.
			            Ensuring efficient file distribution, scalability, and fault tolerance, replicating the core functionalities of traditional
			            torrent clients and trackers.
			            \newline \faLink\ \href{https://github.com/aldairlfp/DTorrent}{https://github.com/aldairlfp/DTorrent}}
		\end{cvsectionnormaltext}}

	\vspace{-3mm}
	\cventry
	{}
	{LDAPSphere \vspace{-5mm}}
	{Python \vspace{-5mm}}
	{}
	{\begin{cvsectionnormaltext}
			\item{LDAPSphere is a distributed directory service that provides a unified interface for managing user identities and access control across multiple platforms.
			            It features a modular architecture, allowing for easy integration with existing systems and support for various authentication methods.
			            \newline \faLink\ \href{https://github.com/aldairlfp/LDAPSphere}{https://github.com/aldairlfp/LDAPSphere}}
		\end{cvsectionnormaltext}}

	\vspace{-5mm}

\end{cventries}

%%%%%%%%%%%%%%%%%%%%%%%%%%%%%%%%%%%%%%
%     Education
%%%%%%%%%%%%%%%%%%%%%%%%%%%%%%%%%%%%%%
\vspace{8mm}
\cvsection{Education}
\begin{cventries}
	\vspace{-3mm}
	\cventry
	{}
	{University of Havana \vspace{-5mm}}
	{Havana, Cuba \vspace{-5mm}}
	{09/2018 – 03/2025 \vspace{-5mm}}
	{\begin{cvsectionnormaltext}
			\item{BSc in Computer Science (7.8/10)}
		\end{cvsectionnormaltext}}
\end{cventries}

\end{document}